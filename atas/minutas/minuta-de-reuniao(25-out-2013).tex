%---------------------------------------------------------------------
\subsubsection{Minuta de reuni�o (25-out-2013)}

\begin{tabbing}
  Local \= xxx \kill
  Local \> : LEAD \\
  Data  \> : 25 de Outubro de 2013 \\
  Hora  \> : 13:00
\end{tabbing}

%---------------------------------------------------------------------
\participantes{
  \jacoud,
  \andre,
  \elael,
  \julia,
  \patrick,
  \rafael,
  \ramon,
  \renan.
}

\begin{itemize}
  \item Update semanal. Resumo do que cada um estudou, as restri��es/recomenda��es e tarefas para a pr�xima semana.

  \begin{itemize}
    \item \textbf{\andre.} Estudou baterias e isolamento de cabos.
    Recomenda-se estudar sistemas de gerenciamento de pot�ncia e sincroniza��o dos equipamentos (time stamping).
    Pesquisar em ROV's: Sistemas de alimenta��o e umbilical. \\
    Nova Tarefa: Fazer um apanhado de possibilidades de sistemas de pot�ncia e umbilicais para saber qual se encaixaria melhor no projeto. \\

    \item \textbf{\rafael.} Novo bolsista de Mestrado. Tarefa preliminar: Explorar RockRobotics, familiarizar-se com a linguagem usada no projeto. \\
        Nova Tarefa: Instalar o ROCK, entender/familiarizar-se  com a programa��o do software e tentar resolver o primeiro exemplo do site. \\

    \item \textbf{\julia.} Trabalhando com identidade Visual, Planilha de aluguel Sylvain/ Invent�rio do Laborat�rio. \\
        Procedimentos de compras para o laborat�rio. \\
        Nova Tarefa: Site do projeto. Estrutura (perguntas, modelo, necessidades x usu�rios). Pesquisar sobre ROCK/Stoplogs, Pack Interface. \\
        Lembrete para Ramon: Contactar a acessoria de imprensa para divulgar o evento com a SBR em 3 de Nov. \\

    \item \textbf{\renan.} Foco em sensores de For�a. Fez pesquisa acessoria de forca, como os sensores funcionam, m�tricas importantes de mercado e o tipo de sensores que poder�amos usar no projeto. \\
        Nova Tarefa: Resumo de pros \& cons de sensores magn�ticos, focar em sensors a prova d'�gua.  Levantamento dos poss�veis m�todos para contato. Mini apresenta��o para ser discutida na semana que vem. \\

    \item \textbf{\elael.} Definido entre software e electr�nica, recomenda-se integra-lo no time de software que j� esta formado no LEAD. Estudou a documenta��o do ROCK.  \\
        Nova Tarefa: instalar o ROCK, entender/se familiarizar com a programa��o de software e tentar resolver o primeiro exemplo do site. Criar um drive. \\
  \end{itemize}

\end{itemize}

\vspace{10mm}%
\parbox[t]{70mm}{
  Aprovado por: \\[5mm]
  \centering
  %\includegraphics[bb=1 1 1238 299,width=65mm]{../assinatura/assinatura-digital.jpg} \\[-4mm]
  \includegraphics[width=65mm]{../assinatura/assinatura-digital.jpg} \\[-4mm]
  \rule[2mm]{70mm}{0.1mm} \\
  \ramon \\[1mm]
  Coordenador do Projeto \\
}

%---------------------------------------------------------------------
\fim


