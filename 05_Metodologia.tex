
%%******************************************************************************
%% SECTION - Descrição sumária da metodologia adotada no desenvolvimento das etapas;
%%******************************************************************************
\setcounter{secnumdepth}{3}
\section{Metodologia}
\label{metodologia}

\subsection{Definição}  
Após a análise e compreensão do problema de inserção e remoção de Stoplogs em visita de campo, foi feita uma pesquisa
\textbf{Pesquisa Bibliográfica} e \emph{brainstorm} com o objetivo de alcançar um conceito sólido de solução ao problema. 

A partir do resultado dessa pesquisa foi desenvolvido um conceito base de solução robótica, descrito na seção {\bf Escopo} do projeto básico. Baseado neste conceito, foram realizadas pesquisas de tecnologias e de fornecedores (secção  {\bf Pesquisa Tecnológica} do projeto básico) de forma recursiva e convergente com relação aos resultados. Isto
é, com base nas pesquisas de solução tecnológicas possíveis, buscam-se fornecedores compatíveis e com o
resultado e informação dos produtos dos fornecedores encontrados faz-se
novamente uma pesquisa de tecnologia	, agora mais aprofundada, e assim sucessivamente
até encontrar-se um resultado final satisfatório. 

Esta pesquisa já é focada nos
componentes a serem utilizados, dessa maneira, os fornecedores escolhidos eram
baseados não somente na conformidade técnica, mas também tempo de entrega,
dificuldade de importação, suporte e reconhecimento. O escopo inicial de solução
é então atualizado e detalhado de acordo com o resultado desta pesquisa, resultando na descrição do robô a ser construído no projeto.

\subsection{Execução}
Na segunda etapa as definições realizada na primeira etapa são executadas: 
Os materiais especificados foram requisitados pelo administrativo do projeto ao respectivos fabricantes definidos.
Os algoritmos definidos foram implementadas no framework de robótica do projeto pela equipe de desenvolvimento.
A interface gráfica foi prototipada e a compreensão do fluxo de informação testada com o usuário final.  

